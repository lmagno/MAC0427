\documentclass[a4paper,11pt]{article}
\usepackage[T1]{fontenc}
\usepackage[utf8]{inputenc}
\usepackage{graphicx}
\usepackage{amsmath}
\usepackage[left=2.5cm,right=2.5cm,top=2.0cm,bottom=1.5cm]{geometry}
\usepackage[hidelinks]{hyperref}
\usepackage{indentfirst}
\usepackage{caption}
\usepackage{subcaption}
\usepackage{algorithm}
\usepackage{algpseudocode}
\usepackage{txfonts} % Para usar o símbolo dos números reais (\varmathbb{R})
\usepackage[brazilian]{babel}
\usepackage{natbib}
\bibliographystyle{plainnat}
\usepackage{enumitem}
\setlist[description]{leftmargin=\parindent,labelindent=\parindent}

\date{}
\author{Lucas Magno \\ 7994983}
\title{Programação Não-Linear \\ Algoritmos de Busca Linear}

\begin{document}
    \maketitle

    \section*{Introdução}
    \section*{Otimização Irrestrita}
        Dado $f: \varmathbb{R}^n \rightarrow \varmathbb{R}$, um problema de
        otimização irrestrita pode ser escrito
        \begin{equation*}
            \begin{aligned}
                & \text{minimizar} & & f(x) \\
                & \text{sujeito a} & & x \in \varmathbb{R}^n \\
            \end{aligned}
        \end{equation*}
        ou seja, queremos encontrar $x^*$ que minimize $f$ (dito minimizador), o que pode se dar em duas formas:
        \begin{description}
            \item [Minimizador global] $f(x^*) \leq f(x) \quad  \forall x \in \varmathbb{R}^n$
            \item [Minimizador local]  $\,\,f(x^*) \leq f(x) \quad  \forall x \mid \|x - x^*\| \leq \epsilon$
                    para algum $\epsilon > 0$ (isto é, existe uma vizinhança na qual $f(x^*)$ tem valor mínimo).
        \end{description}

        Analogamente a problemas unidimensionais, sendo $f$ diferenciável, pode-se mostrar que, se $x^*$ é um minimizador
        (local ou global), então
            $$ \nabla f(x^*) = 0 $$
        o que nos dá uma forma de encontrar os minimizadores, pois basta encontrar os pontos que anulam o gradiente (dito estacionários).
        No entanto, de forma geral, não temos como determinar localmente se um ponto é mínimo global, então os métodos implementados se contentarão
        em encontrar mínimos locais (sejam eles globais ou não).

    \newpage
    \section*{Busca Linear}
        Os métodos utilizados neste trabalho seguem todos o mesmo paradigma: a \emph{busca linear}.
        Tal paradigma consiste basicamente em se escolher, a partir de um ponto $x$, uma direção de busca $d$ e então escolher
        um passo $\alpha$ nessa direção, de forma que o valor da função $f$ que queremos
        minimizar diminua adequadamente (cujo significado será discutido adiante).

        Ou seja, a forma básica da busca linear é (dado $x^0$):
        \begin{algorithm}[h]
            \caption{Busca Linear}
            \label{alg:ls}
            \begin{algorithmic}[1]
                \State $k \gets 0$
                \While {$\nabla f(x^k) \neq 0$}
                    \State Escolher $d^k \in \varmathbb{R}^n$ tal que $\nabla f(x^k)^\top d^k < 0$
                    \State Escolher $\alpha_k \in \varmathbb{R}$\,\, tal que $f(x^k + \alpha_k d^k) < f(x^k)$
                    \State $x^{k+1} \gets x^k + \alpha_k d^k$
                    \State $k \gets k + 1$
                \EndWhile
            \end{algorithmic}
        \end{algorithm}

        Em que a linha 4 impõe o decréscimo da função e a 5 é a atualização da iteração.
        A linha 3, entretanto, introduz a condição
        \begin{equation}
            \label{eq:desc}
            \nabla f(x^k)^\top d^k < 0
        \end{equation}
        cuja motivação pode ser vista através da expansão em Taylor da função:
            $$ f(x + \alpha d) = f(x) + \alpha\nabla f(x)^\top d + \frac{1}{2}\alpha^2d^\top\nabla^2 f(x)d + o(\|\alpha d\|^2)  $$
        assim, sempre podemos escolher um passo $\alpha$ suficientemente pequeno tal que o termo de primeira ordem domine os seguintes
        e, por ele ser negativo (restringindo $\alpha > 0$), vale que
            $$ f(x + \alpha d) < f(x) $$

        Portanto, direções que satisfaçam essa condição garantem que seja sempre possível descrescer o valor da função e por isso
        são chamadas de \emph{direções de descida}.

        Esse algoritmo gera a sequência
            $$ \{x^k\}_{k = 0}^\infty $$
        para a qual \emph{esperaríamos} que valesse
            $$ \underset{k \rightarrow \infty}{lim} x^k = x^*$$
        mas não é o caso, pois, dados
            \begin{align*}
                f(x) &= {x}^2 \\
                x^k  &= 1 + \frac{1}{k} \\
                d^k  &= -1
            \end{align*}
        para todo $k > 0$ e notando que $x^{k+1} < x^{k}$, temos
        \begin{align*}
            \alpha_k &= x^k - x^{k+1}  & &> 0 \\
            \nabla f(x^k)^\top d^k &= -2x^k & &< 0 \quad \forall x > 0 \\
            f(x^{k+1}) - f(x^k) &= (x^{k+1})^2 - (x^{k})^2 & &< 0
        \end{align*}

        Logo, essa é uma sequência perfeitamente válida ao que concerne o algoritmo \ref{alg:ls} e apesar disso é fácil notar que
            $$ \underset{k \rightarrow \infty}{lim} x^k = 1$$
        que não é nem ponto estacionário de $f$, então essas condições não são suficientes para garantir a convergência do algoritmo.
    \section*{Condições}
        Felizmente, com apenas algumas condições a mais podemos resolver o problema da convergência, a saber:

        \subsection*{Condição de Armijo}
            É natural que exijamos que a função deva decrescer a cada iteração, mas somente isso basta? \emph{Quanto} $f$
            deve decrescer? Uma forma de exigir um decréscimo mínimo é através da \emph{condição de Armijo}:

            Dado $0 < \gamma < 1$ (normalmente $\gamma = 10^{-4}$)
                $$ f(x + \alpha d) < f(x) + \gamma \alpha \nabla f(x)^\top d $$
            que, sendo $d$ direção de descida (vale a equação \ref{eq:desc}), impõe um limite superior no valor da função
            no novo ponto melhor do que a condição anterior (a qual equivale a fazermos $\gamma = 0$ aqui).

            Além disso, sempre existe um $\alpha$ suficientemente pequeno para o qual Armijo vale, então podemos utilizar
            a seguinte técnica, conhecida como \emph{backtracking} para determinar um passo satisfatório: partindo de um
            passo candidato, se este não satisfizer Armijo, diminua ele e tente novamente. Repetindo essas etapas, chegaremos a
            um valor que o satisfaça e então o adotamos.

        \subsection*{Condição do Passo}
            É importante, entretanto, que esses passos não sejam muito pequenos, pois, se estes diminuirem rápido demais, a sequência pode
            convergir a algum ponto de acumulação que não tem propriedade de mínimo ou ponto estacionário. É o que acontece
            no exemplo anterior, em que os passos $\alpha_k$ convergem rapidamente para zero, levando o algoritmo a um ponto
            qualquer (no ponto de vista da minimização).

            De forma similar, queremos evitar passos muito longos, pois o algoritmo pode ficar preso num zigue-zague ao redor do
            ponto estacionário.

            Podemos evitar esses problemas, então, limitando o valor de um novo passo com base no anterior. Por exemplo:
            $$ 0.1\alpha_k \leq \alpha_{k+1} \leq 0.9\alpha_k$$

            E como escolher o valor do passo? Um jeito simples e inteligente é usando interpolação polinomial nos pontos
            anteriores, podendo aproveitar os valores da função e seu gradiente já calculados. Neste trabalho foram utilizadas
            a interpolação quadrática e a cúbica para comparação.

        \subsection*{Condição da Norma}
            Ainda assim, como o avanço em cada iteração
                $$x_{k+1} - x_k = \alpha_k d^k$$
            depende tanto do passo quanto da direção, valem as mesmas observações do item anterior sobre a norma de $d^k$, portanto
            também queremos restringir ela e podemos exigir que:
                $$ \|d^k\| \geq \sigma \|\nabla f(x^k) \|$$
            para algum $\sigma > 0$.

            Assim, limitamos inferiormente o avanço e mesmo assim permitindo que este seja suficientemente pequeno perto do ponto estacionário
            ($ \nabla f $ pequeno).

        \subsection*{Condição do Ângulo}
            Outro problema é a própria direção de busca. Analogamente ao tamanho do passo, apenas exigir que ela seja direção de descida não
            é suficiente, pois ela pode ser arbitrariamente próxima à ortogonalidade com o gradiente, apontado ``para o lado'', ao invés de
            para o ponto estacionário, fazendo com que o avanço na direção deste seja desprezível.

            Então podemos simplesmente exigir que $d$ faça um ângulo máximo com o sentido de $-\nabla f$, isto é (através da relação entre cosseno
            e produto interno):
                $$ \nabla f(x)^\top d \leq -\theta \|\nabla f(x)\| \|d\| $$
            para algum $0 < \theta \leq 1$.

    \section*{Métodos}
        Como visto, a busca linear é um paradigma razoavelmente simples, mas como escolhemos a direção de busca? Há várias estratégias para isso e é
        isso que diferencia os diversos métodos existentes.

        \subsection*{Método do Gradiente}
            Talvez o método mais simples, consiste em se escolher a direção de busca como sendo no sentido oposto ao gradiente:
                $$ d = -\nabla f(x) $$

            Então, ao menos localmente, $d$ aponta para onde a função mais descresce e por isso este método também é conhecido como \emph{máxima descida.}

            É fácil verificar que esta direção satisfaz as condições de norma (com $\sigma \leq 1$) e ângulo, então só é necessário verificar Armijo e o tamanho do passo.

            Esta simplicidade, entretanto, tem suas desvantagens: apesar de ter convergência global (para qualquer ponto inicial) garantida,
            por sua direção ser sempre de descida, a taxa dessa é apenas linear. Pior ainda, problemas com diferenças de escala muito grandes dificultam a convergência
            e o método pode se tornar ineficiente.

        \subsection*{Método de Newton}
            Mais complexo que o método do gradiente, o método de Newton se baseia em pegar a direção que minimiza uma aproximação quadrática da função objetiva,
            isto é, os primeiros termos de expansão de $f$ em série de Taylor:
                $$ f(x + d) = f(x) + \nabla f(x)^\top d + \frac{1}{2}d^\top\nabla^2f(x)d $$
            e é simples ver que tal direção é solução do sistema
                $$ \nabla^2 f(x)d = -\nabla f(x)$$

            Por utilizar informações de segunda ordem sobre a função, podemos esperar um melhor comportamento para este método, comparando com o anterior.
            De fato, o método de Newton não apresenta problemas com escalas e no geral sua convergência é quadrática, ao menos para pontos suficientemente
            próximos da solução.

            Por outro lado, ele exige o cálculo da hessiana a cada iteração e ainda a resolução de um sistema linear, então não é um método barato.
            Além disso, a hessiana pode ser singular, já que não temos restrição nenhuma sobre ela, e logo não existir solução para o sistema,
            ou então esta pode não satisfazer a condição do ângulo\footnote{Pode não satisfazer a condição da norma também, mas isso exigiria apenas uma normalização e não o cálculo de uma nova direção.}.

            Nesses casos, porém, podemos utilizar no sistema uma forma alterada da hessiana:
                $$ B = \nabla^2 f(x) + \rho I $$
            com $\rho > 0$, o que é chamado de \emph{shift} nos autovalores, pois cada autovalor da hessiana terá o valor $\rho$ somado a ele.

            Assim, para $\rho$ suficientemente grande, todos os autovalores serão positivos e portanto $B$ será definida positiva, o que já garante
            a existência de uma solução e que esta será uma direção de descida. Melhor ainda, note que quanto maior a constante, mais $B$ se aproxima
            da identidade e portanto do método do gradiente, o qual sabemos que satisfaz a condição do ângulo.

        \subsection*{Método BFGS}
            A fim de ser menos custoso que o método de Newton, mas mais esperto que o de máxima descida, o método BFGS\footnote{Broyden–Fletcher–Goldfarb–Shanno.}
            é análogo ao método de secantes para se achar zero de funções, no sentido de que constrói uma aproximação da hessiana, de segunda ordem, a partir
            de diversas informações de primeira ordem.

            Dessa forma, não é necessário o cálculo da hessiana a cada iteração, apenas uma aproximação, e ainda podemos exigir que tal aproximação
            a cada ponto seja uma atualização da anterior, exigindo menos cálculos ainda. Mais ainda, podemos aproximar diretamente a \emph{inversa}
            da hessiana e eliminar de vez a necessidade de se resolver um sistema linear.

            Essas exigências, mais o fato de que essas aproximações sejam definidas positivas, determinam o método BFGS, que, embora não seja
            simples mostrar, consiste em calcular, a cada iteração
                $$ H_{k+1} = H_k + \left( \frac{1 + q_k^\top H_kq_k}{p_k^\top q_k}\right)\frac{p_kp_k^\top}{p_k^\top q_k} - \frac{p_kq_k^\top H_k + H_kq_kp_k^\top}{p_k^\top q_k}$$
            em que\footnote{Vale notar que $H_k$ também deve satisfazer $H_kq_j = p_j$ para todo $j \leq k$, que, não por coincidência, é a chamada \emph{equação secante}.}

                \begin{equation*}
                    \begin{aligned}
                        p_k &= x^{k+1} & &- x^k \\
                        q_k &= \nabla f(x^{k+1}) & &- \nabla f(x^k) \\
                    \end{aligned}
                \end{equation*}
            e então, na próxima iteração, calcular
                $$ d_k = -H_k\nabla f(x^k) $$

            Como $H_k$ é dada por uma fórmula iterativa, é necessário definir um valor inicial $H_0$, contanto que seja uma matriz definida positiva, para manter as propriedades desejadas. Escolhas comuns são a matriz identidade (simples, mas nenhuma relação com a hessiana) e a própria hessiana (cujo cálculo pode ser caro).

            Como no método de Newton, a direção de busca calculada pode não satisfazer a condição do ângulo e poderíamos lidar com isso similarmente. Mas, como temos liberdade de se afastar da hessiana, podemos utilizar a matriz identidade como novo $H$ e repetir o cálculo de $d$, o que efetivamente é o método de gradiente e portanto satisfaz a condição do ângulo.

            Devido a isso e à simplicidade, a identidade é uma escolha razoável para $H_0$.

            Novamente de forma análoga ao método das secantes, a convergência do BFGS fica entre os métodos de primeira e segunda ordem, isto é, é superlinear.

    \section*{Implementação}
        Para testar os métodos foi implementado um programa em \texttt{FORTRAN} que utiliza o conjunto de  funções implementadas em \citep*{More:1981:AFS:355934.355943}.
        O programa em si é divido em vários módulos separados, a saber:

        \begin{description}
            \item[mgh] Faz a interface com os códigos em \texttt{F77} de \citep*{More:1981:AFS:355934.355943}, fornecendo a função objetiva, gradiente e hessiana para os 18 problemas implementados.
            \item[ls] Contém os algoritmos que realizam a escolha do passo $\alpha$ através da busca linear por interpolação quadrática (\texttt{lsquad}) e cúbica (\texttt{lscube}).
            \item[]
            \item[]
            \item[]
            \item[]
            \item[]

        \end{description}
    \section*{Resultados}
    \section*{Conclusão}

    \bibliography{EP1}
\end{document}
