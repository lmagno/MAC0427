\documentclass[a4paper,11pt]{article}
\usepackage[T1]{fontenc}
\usepackage[utf8]{inputenc}
\usepackage{graphicx}
\usepackage{amsmath}
\usepackage[left=2.5cm,right=2.5cm,top=2.0cm,bottom=1.5cm]{geometry}
\usepackage[hidelinks]{hyperref}
\usepackage{indentfirst}
\usepackage{caption}
\usepackage{subcaption}
\usepackage{algorithm}
\usepackage{algpseudocode}
\usepackage{txfonts} % Para usar o símbolo dos números reais (\varmathbb{R})
\usepackage[brazilian]{babel}
\usepackage{natbib}
\bibliographystyle{plainnat}
\usepackage{enumitem}
\usepackage{dirtree}
\setlist[description]{leftmargin=2\parindent,labelindent=2\parindent}

\date{}
\author{Lucas Magno \\ 7994983}
\title{Programação Não-Linear \\ Algoritmos de Busca Linear}

\begin{document}
    \maketitle

    \section*{Introdução}
        Seja uma pessoa querendo controlar gastos ou um cientista buscando o estado de menor energia de um sistema, a miniminização (e seu equivalente, maximimização) é um problema ubíquo e não é surpreendente que pessoas buscassem desenvolver técnicas sofisticadas para sua resolução.

        Essa classe de problemas, conhecida como otimização, apresenta diversas abordagens diferentes e neste trabalho será explorada uma delas: a otimização contínua irrestrita.
        Isto é, focaremos na minimização de funções contínuas, sem se preocupar em que região do seu domínio as soluções podem estar.

        Mais especificadamente, exploraremos o paradigma de busca linear, um dos mais conhecidos juntamente com o de região de confiança (que não será tratado aqui).
        Com isso se espera determinar a eficiência dos algoritmos mais conhecidos e em que contexto eles são melhor aplicados.

    \section*{Otimização Irrestrita}
        Dado $f: \varmathbb{R}^n \rightarrow \varmathbb{R}$, um problema de
        otimização irrestrita pode ser escrito
        \begin{equation*}
            \begin{aligned}
                & \text{minimizar} & & f(x) \\
                & \text{sujeito a} & & x \in \varmathbb{R}^n \\
            \end{aligned}
        \end{equation*}
        ou seja, queremos encontrar $x^*$ que minimize $f$ (dito minimizador), o que pode se dar em duas formas:
        \begin{description}
            \item [Minimizador global] $f(x^*) \leq f(x) \quad  \forall x \in \varmathbb{R}^n$
            \item [Minimizador local]  $\,\,f(x^*) \leq f(x) \quad  \forall x \mid \|x - x^*\| \leq \epsilon$
                    para algum $\epsilon > 0$ (isto é, existe uma vizinhança na qual $f(x^*)$ tem valor mínimo).
        \end{description}

        Analogamente a problemas unidimensionais, sendo $f$ diferenciável, pode-se mostrar \citep*{Nocedal2006NO} que, se $x^*$ é um minimizador
        (local ou global), então
            $$ \nabla f(x^*) = 0 $$
        o que nos dá uma forma de encontrar os minimizadores, pois basta encontrar os pontos que anulam o gradiente (dito estacionários).
        No entanto, de forma geral, não temos como determinar localmente se um ponto é mínimo global, então os métodos implementados se contentarão
        em encontrar mínimos locais (sejam eles globais ou não).

    \newpage
    \section*{Busca Linear}
        Os métodos utilizados neste trabalho seguem todos o mesmo paradigma: a \emph{busca linear}.
        Tal paradigma consiste basicamente em se escolher, a partir de um ponto $x$, uma direção de busca $d$ e então escolher
        um passo $\alpha$ nessa direção, de forma que o valor da função $f$ que queremos
        minimizar diminua adequadamente (cujo significado será discutido adiante).

        Ou seja, a forma básica da busca linear é (dado $x^0$):
        \begin{algorithm}[h]
            \caption{Busca Linear}
            \label{alg:ls}
            \begin{algorithmic}[1]
                \State $k \gets 0$
                \While {$\nabla f(x^k) \neq 0$}
                    \State Escolher $d^k \in \varmathbb{R}^n$ tal que $\nabla f(x^k)^\top d^k < 0$
                    \State Escolher $\alpha_k \in \varmathbb{R}$\,\, tal que $f(x^k + \alpha_k d^k) < f(x^k)$
                    \State $x^{k+1} \gets x^k + \alpha_k d^k$
                    \State $k \gets k + 1$
                \EndWhile
            \end{algorithmic}
        \end{algorithm}

        Em que a linha 4 impõe o decréscimo da função e a 5 é a atualização da iteração.
        A linha 3, entretanto, introduz a condição
        \begin{equation}
            \label{eq:desc}
            \nabla f(x^k)^\top d^k < 0
        \end{equation}
        cuja motivação pode ser vista através da expansão em Taylor da função:
            $$ f(x + \alpha d) = f(x) + \alpha\nabla f(x)^\top d + \frac{1}{2}\alpha^2d^\top\nabla^2 f(x)d + o(\|\alpha d\|^2)  $$
        assim, sempre podemos escolher um passo $\alpha$ suficientemente pequeno tal que o termo de primeira ordem domine os seguintes
        e, por ele ser negativo (restringindo $\alpha > 0$), vale que
            $$ f(x + \alpha d) < f(x) $$

        Portanto, direções que satisfaçam essa condição garantem que seja sempre possível descrescer o valor da função e por isso
        são chamadas de \emph{direções de descida}.

        Esse algoritmo gera a sequência
            $$ \{x^k\}_{k = 0}^\infty $$
        para a qual \emph{esperaríamos} que valesse
            $$ \underset{k \rightarrow \infty}{lim} x^k = x^*$$
        mas não é o caso, pois, dados
            \begin{align*}
                f(x) &= {x}^2 \\
                x^k  &= 1 + \frac{1}{k} \\
                d^k  &= -1
            \end{align*}
        para todo $k > 0$ e notando que $x^{k+1} < x^{k}$, temos
        \begin{align*}
            \alpha_k &= x^k - x^{k+1}  & &> 0 \\
            \nabla f(x^k)^\top d^k &= -2x^k & &< 0 \quad \forall x > 0 \\
            f(x^{k+1}) - f(x^k) &= (x^{k+1})^2 - (x^{k})^2 & &< 0
        \end{align*}

        Logo, essa é uma sequência perfeitamente válida ao que concerne o algoritmo \ref{alg:ls} e apesar disso é fácil notar que
            $$ \underset{k \rightarrow \infty}{lim} x^k = 1$$
        que não é nem ponto estacionário de $f$, então essas condições não são suficientes para garantir a convergência do algoritmo.
    \section*{Condições}
        Felizmente, com apenas algumas condições a mais podemos resolver o problema da convergência, a saber:

        \subsection*{Condição de Armijo}
            É natural que exijamos que a função deva decrescer a cada iteração, mas somente isso basta? \emph{Quanto} $f$
            deve decrescer? Uma forma de exigir um decréscimo mínimo é através da \emph{condição de Armijo}:

            Dado $0 < \gamma < 1$ (normalmente $\gamma = 10^{-4}$)
                $$ f(x + \alpha d) \leq f(x) + \gamma \alpha \nabla f(x)^\top d $$
            que, sendo $d$ direção de descida (vale a equação \ref{eq:desc}), impõe um limite superior no valor da função
            no novo ponto de forma mais exigente do que a condição anterior (a qual equivale à escolha $\gamma = 0$).

            Além disso, sempre existe um $\alpha$ suficientemente pequeno para o qual Armijo vale, o que sugere
            a seguinte técnica, conhecida como \emph{backtracking} para determinar um passo satisfatório: partindo de um
            passo candidato\footnote{É importante que esse passo candidato seja $\alpha = 1$ para os métodos de Newton e Quasi-Newton, pois para pontos suficientemente próximos da solução ele é sempre aceito e isso que dá origem a convergência quadrática e superlinear dos métodos, respectivamente.}, se este não satisfizer Armijo, diminua ele e tente novamente. Repetindo essas etapas, chegaremos a
            um valor que o satisfaça e então o adotamos.

        \subsection*{Condição do Passo}
            É importante, entretanto, que esses passos não sejam muito pequenos, pois, se estes diminuirem rápido demais, a sequência pode
            convergir a algum ponto de acumulação que não tem propriedade de mínimo ou ponto estacionário. É o que acontece
            no exemplo anterior, em que os passos $\alpha_k$ convergem rapidamente para zero, levando o algoritmo a um ponto
            qualquer (no ponto de vista da minimização).

            De forma similar, queremos evitar passos muito longos, pois o algoritmo pode ficar preso num zigue-zague ao redor do
            ponto estacionário.

            Podemos evitar esses problemas, então, limitando o valor de um novo passo com base no anterior. Por exemplo:
            $$ 0.1\alpha_k \leq \alpha_{k+1} \leq 0.9\alpha_k$$

            E como escolher o valor do passo? Um jeito simples e inteligente é usando interpolação polinomial nos pontos
            anteriores, podendo aproveitar os valores da função e seu gradiente já calculados. Neste trabalho foram utilizadas
            a interpolação quadrática e a cúbica para comparação.

        \subsection*{Condição da Norma}
            Ainda assim, como o avanço em cada iteração
                $$x_{k+1} - x_k = \alpha_k d^k$$
            depende tanto do passo quanto da direção, valem as mesmas observações do item anterior sobre a norma de $d^k$, portanto
            também queremos restringir ela e podemos exigir que:
                $$ \|d^k\| \geq \sigma \|\nabla f(x^k) \|$$
            para algum $\sigma > 0$.

            Limitando, portanto, inferiormente o avanço e mesmo assim permitindo que este seja suficientemente pequeno perto do ponto estacionário
            ($ \nabla f $ pequeno).

        \subsection*{Condição do Ângulo}
            Outro problema é a própria direção de busca. Analogamente ao tamanho do passo, apenas exigir que ela seja direção de descida não
            é suficiente, pois ela pode ser arbitrariamente próxima à ortogonalidade com o gradiente, apontado ``para o lado'', ao invés de
            para o ponto estacionário, fazendo com que o avanço na direção deste seja desprezível.

            Então podemos simplesmente exigir que $d$ faça um ângulo máximo com o sentido de $-\nabla f$, isto é (através da relação entre cosseno
            e produto interno):
                $$ \nabla f(x)^\top d \leq -\theta \|\nabla f(x)\| \|d\| $$
            para algum $0 < \theta \leq 1$.

    \newpage
    \section*{Métodos}
        Como visto, a busca linear é um paradigma razoavelmente simples, mas como escolhemos a direção de busca? Há várias estratégias para tal e é
        isso que diferencia os diversos métodos existentes.

        \subsection*{Método do Gradiente}
            Talvez o método mais simples, consiste em se escolher a direção de busca como sendo no sentido oposto ao gradiente:
                $$ d = -\nabla f(x) $$

            Então, ao menos localmente, $d$ aponta para onde a função mais descresce, o que dá ao método o nome de \emph{máxima descida.}

            É fácil verificar que esta direção satisfaz as condições de norma (com $\sigma \leq 1$) e ângulo, então só é necessário verificar Armijo e o tamanho do passo.

            Esta simplicidade, entretanto, tem suas desvantagens: apesar de ter convergência global (para qualquer ponto inicial) garantida,
            por sua direção ser sempre de descida, a taxa dessa é apenas linear. Pior ainda, problemas com diferenças de escala muito grandes dificultam a convergência
            e o método pode se tornar ineficiente.

        \subsection*{Método de Newton}
            Mais complexo que o método do gradiente, o método de Newton se baseia em pegar a direção que minimiza uma aproximação quadrática da função objetiva,
            isto é, os primeiros termos de expansão de $f$ em série de Taylor:
                $$ f(x + d) = f(x) + \nabla f(x)^\top d + \frac{1}{2}d^\top\nabla^2f(x)d $$
            e é simples ver que tal direção é solução do sistema\footnote{Portanto o método de Newton exige $f \in C^2$.}
                $$ \nabla^2 f(x)d = -\nabla f(x)$$

            Por utilizar informações de segunda ordem sobre a função, podemos esperar um melhor comportamento para este método, comparando com o anterior.
            De fato, o método de Newton não apresenta problemas com escalas e no geral sua convergência é quadrática, ao menos para pontos suficientemente
            próximos da solução.

            Por outro lado, ele exige o cálculo da hessiana a cada iteração e ainda a resolução de um sistema linear, então não é um método barato.
            Além disso, a hessiana pode ser singular, já que não temos restrição nenhuma sobre ela, e logo não existir solução para o sistema,
            ou então esta pode não satisfazer a condição do ângulo\footnote{Pode não satisfazer a condição da norma também, mas isso exigiria apenas uma normalização e não o cálculo de uma nova direção.}.

            Nesses casos, porém, podemos utilizar no sistema uma forma alterada da hessiana:
                $$ B = \nabla^2 f(x) + \rho I $$
            com $\rho > 0$, o que é chamado de \emph{shift} nos autovalores, pois cada autovalor da hessiana terá o valor $\rho$ somado a ele.

            Assim, para $\rho$ suficientemente grande, todos os autovalores serão positivos e portanto $B$ será definida positiva, o que já garante
            a existência de uma solução e que esta será uma direção de descida. Melhor ainda, note que quanto maior a constante, mais $B$ se aproxima
            da identidade e portanto do método do gradiente, o qual sabemos que satisfaz a condição do ângulo.

        \subsection*{Método BFGS (Quasi-Newton)}
            A fim de ser menos custoso que o método de Newton, mas mais esperto que o de máxima descida, o método BFGS\footnote{Broyden–Fletcher–Goldfarb–Shanno.} (um dos métodos conhecidos como \emph{Quasi-Newton}) é análogo ao método de secantes para se achar zero de funções, no sentido de que constrói uma aproximação da hessiana, de segunda ordem, a partir
            de diversas informações de primeira ordem.

            Dessa forma, não é necessário o cálculo da hessiana a cada iteração, apenas uma aproximação, e ainda podemos exigir que tal aproximação
            a cada ponto seja uma atualização da anterior, exigindo menos cálculos ainda. Mais ainda, podemos aproximar diretamente a \emph{inversa}
            da hessiana e eliminar de vez a necessidade de se resolver um sistema linear.

            Essas exigências, mais o fato de que essas aproximações sejam definidas positivas, determinam o método BFGS, que, embora não seja
            simples mostrar \citep*{Friedlander94} \footnote{Embora a referência tenha sido o livro da Ana, na Wikipédia (\href{https://en.wikipedia.org/wiki/Broyden-Fletcher-Goldfarb-Shanno_algorithm}{BFGS}) encontrei essa expressão com o fator $p_k^\top q_k$ ao invés da constante $1$ e, na prática, o algoritmo se mostrou muito mais eficiente nesse caso, então escolhi essa fórmula.}, consiste em calcular, a cada iteração
                $$ H_{k+1} = H_k + \left( \frac{p_k^\top q_k + q_k^\top H_kq_k}{p_k^\top q_k}\right)\frac{p_kp_k^\top}{p_k^\top q_k} - \frac{p_kq_k^\top H_k + H_kq_kp_k^\top}{p_k^\top q_k}$$
            em que\footnote{Vale notar que $H_k$ também deve satisfazer $H_kq_j = p_j$ para todo $j \leq k$, que, não por coincidência, é a chamada \emph{equação secante}.}

                \begin{equation*}
                    \begin{aligned}
                        p_k &= x^{k+1} & &- x^k \\
                        q_k &= \nabla f(x^{k+1}) & &- \nabla f(x^k) \\
                    \end{aligned}
                \end{equation*}
            e então, na próxima iteração, calcular
                $$ d_k = -H_k\nabla f(x^k) $$

            Como $H_k$ é dada por uma fórmula iterativa, é necessário definir um valor inicial $H_0$, contanto que seja uma matriz definida positiva, para manter as propriedades desejadas. Escolhas comuns são a matriz identidade (simples, mas nenhuma relação com a hessiana) e a própria hessiana (cujo cálculo pode ser caro).

            Como no método de Newton, a direção de busca calculada pode não satisfazer a condição do ângulo e poderíamos lidar com isso similarmente. Mas, como temos liberdade de se afastar da hessiana, podemos utilizar a matriz identidade como novo $H$ e repetir o cálculo de $d$, o que efetivamente é o método de gradiente e portanto satisfaz a condição do ângulo.

            Devido a isso e à simplicidade, a identidade é uma escolha razoável para $H_0$.

            Novamente de forma análoga ao método das secantes, a convergência do BFGS fica entre os métodos de primeira e segunda ordem, isto é, é superlinear.

            Vale notar também que essa fórmula para $H_{k+1}$ nos permite a calcular sem o uso de memória auxiliar, já que ela contém escalares ($q_k^\top H_kq_k$, $p_k^\top q_k$),  produtos externos ($p_kp_k^\top$, $p_kq_k^\top$, $q_kp_k^\top$) e seus produtos com $H_k$, que são simples de descrever. A dificuldade se encontra na ordem em que os elementos de $H_{k+1}$ devem ser gravados sobre $H_k$, já que cada elemento daquela depende de todos desta.

            No entanto, observando que a matriz é sempre simétrica (pela fórmula), ou seja, a parte triangular inferior é a transposta da superior, podemos sobrescrever primeiro a parte triangular inferior de $H_k$ (sem a diagonal principal), tomando cuidado para ler somente da triangular superior. Depois, podemos sobrescrever a diagonal principal, novamente lendo somente da parte triangular superior e finalmente basta copiar os valores calculados na triangular inferior para a superior, obtendo a matriz $H_{k+1}$ completa.
    \newpage
    \section*{O Programa}
        Para implementar os métodos foi feito um programa em \texttt{FORTRAN 2008} que testa o conjunto de 18 problemas implementados em \citep*{More:1981:AFS:355934.355943}.

        \subsection*{Módulos}
            O programa em si é divido em vários módulos, a saber:

            \begin{description}
                \item[mgh] Faz a interface com os códigos em \texttt{F77} de MGH, que fornecem as subrotinas (e o que calculam, para cada um dos 18 problemas):
                    \begin{description}
                        \item[\texttt{INITPT}] $x^0$
                        \item[\texttt{OBJFCN}] $f(x)$
                        \item[\texttt{GRDFCN}] $\nabla f(x)$
                        \item[\texttt{HESFCN}] $\nabla^2 f(x)$
                    \end{description}

                \item[ls] Contém os algoritmos que realizam a escolha do passo $\alpha$ através da busca linear por interpolação quadrática (\texttt{lsquad}) e cúbica (\texttt{lscube}).
                \item[methods] Implementa os métodos (\texttt{grad}, \texttt{newt}, \texttt{bfgs}) de fato.
                \item[cholesky, trisys, utils] Conjunto de módulos para a decomposição de matrizes em fatores de Cholesky e a resolução de sistemas lineares que as contenham.
                \item[stats] Fornece \emph{wrappers} e subrotinas para automatizar a contagem e a impressão de diversas etapas durante a execução dos algoritmos.
            \end{description}

        \subsection*{Arquivos}
            No diretório fonte ficam os arquivos:
            \begin{description}
                \item[\texttt{EP1.f08}] Código principal, responsável por juntar os módulos e iniciar a execução.
                \item[\texttt{EP1}] Executável, criado durante a compilação.
                \item[\texttt{EP1.pdf}] Relatório, criado durante a compilação.
                \item[\texttt{Makefile}]
            \end{description}

            Os demais arquivos são distribuídos nos seguintes diretórios:
            \begin{description}
                \item[\texttt{src/}] Contém os respectivos arquivos de cada módulo (\texttt{.f08}), bem como os das subrotinas MGH (\texttt{.f}).
                \item[\texttt{obj/}] Criado durante a compilação, guarda os arquivos objetos (\texttt{.o}) e módulos (\texttt{.mod}). É excluído na limpeza.
                \item[\texttt{tex/}] Contém o código fonte deste relatório (\texttt{.tex}) e a bibliografia (\texttt{.bib}).
                \item[\texttt{aux/}] Criado durante a compilação, guarda os arquivos auxiliares do \LaTeX  e do \texttt{BibTeX}. É excluído na limpeza.
            \end{description}

        \subsection*{Compilação}
            A partir do diretório raiz, os seguintes comandos são possíveis:

            \begin{description}
                \item[make] Compila o programa, criando o executável.
                \item[make tex] Compila o relatório, criando o arquivo \texttt{.pdf}.
                \item[make clean] Realiza a limpeza, deletando os arquivos de saída.
            \end{description}

        \subsection*{Execução}
            Ao executar o programa, para cada um dos problemas, os três métodos apresentados serão testados (e tanto com interpolação quadrática, quanto cúbica).
            Então, para cada problema, 6 processos diferentes são criados e rodados em paralelo, através da chamada para o sistema \texttt{fork()} e o processo principal aguarda o término destes antes de passar para o próximo problema.

            O critério de parada escolhido para cada método foi a norma\footnote{Todas as normas utilizadas nesse trabalho são euclidianas.} do gradiente final, ou seja, fornecido um $\epsilon > 0$, o programa para quando
                $$ \Vert \nabla f(x^k) \Vert < \epsilon $$
            já que desejamos que essa norma seja nula.

            Cada um dos processos, ao terminar, imprime para a tela um resumo dos dados do algorítmo, como norma final do gradiente, número de iterações, chamadas à função objetiva, etc.
            Assim, os resultados de todos os métodos ficam organizados por problema.

            Como exemplo de saída, podemos olhar o seguinte problema:
            \begin{table}[h]
                \centering
                \begin{tabular}{r|rrrrrrrr}
                    1 Helical Valley  & $\Vert\nabla f_*\Vert$ & it & $f$ & $\nabla f $ & $\nabla^2f$ & armijo & norma & ângulo \\
                    \hline
                    newtcube & 0.11E-05  &      12  &      25  &      13   &     12   &      1    &     4   &     50 \\
                    bfgscube & 0.11E-05  &      74  &     263  &      61   &      0   &    143    &     0   &     14 \\
                    bfgsquad & 0.58E-05  &      93  &     327  &      79   &      0   &    171    &     0   &     15 \\
                    newtquad & 0.11E-05  &      12  &      25  &      13   &     12   &      1    &     4   &     50 \\
                    gradquad & 0.98E-05  &    2296  &    4638  &    2297   &      0   &     46    &     0   &      0 \\
                    gradcube & 0.98E-05  &    2287  &    4618  &    2288   &      0   &     44    &     0   &      0 \\
                \end{tabular}
                \caption{Saída do programa para o primeiro problema de Moré-Garbow-Hillstrom (\emph{Helical Valley}), com $\epsilon = \theta = 10^{-5} $, $\gamma = \sigma = 10^{-4}$.}
            \end{table}

            em que cada coluna representa:
            \begin{description}[align=right, leftmargin=4\parindent,labelindent=4\parindent]
                \item [$\Vert\nabla f_*\Vert$] A norma do gradiente no último ponto calculado (solução).
                \item [it] O número de iterações do algorítmo.
                \item [$f$] O número de chamadas da função objetiva realizadas.
                \item [$\nabla f $] O número de chamadas do gradiente realizadas.
                \item [$\nabla^2f$] O número de chamadas da hessiana realizadas.
                \item [armijo] O número de vezes em que a condição de Armijo não foi satisfeita.
                \item [norma] O número de vezes em que a condição da norma não foi satisfeita.
                \item [ângulo] O número de vezes em que a condição do ângulo não foi satisfeita.
            \end{description}

            Além disso, como a execução é em paralelo, não existe uma ordem fixa entre os métodos, isto é, cada um imprime seus resultados assim que terminar. Consequentemente, a ordem final em que os métodos aparecem pode indicar o tempo gasto por cada um. No entanto, questões como o \emph{buffer} para a saída podem alterar essa ordem, então ela não deve ser tomada literalmente e por isso daqui em diante eles serão apresentados em uma ordem fixa, para facilitar a leitura.

            Assim, indicadores melhores do gasto de cada algorítmo são os contadores. O tempo de execução de fato não foi registrado pois esse se mostrou muito abaixo da precisão disponível (milissegundo), não sendo útil para comparação.

    \newpage
    \section*{Resultados}
        Para começar com um exemplo simples, tomemos o parabolóide dado por
            $$ f(x) = x_1^2 + 1000x_2^2 $$
        e cujo mínimo é
            $$ f(0, 0) = 0$$
        e o ponto inicial
            $$ x^0 = \begin{bmatrix}
                        100 \\
                        100
                     \end{bmatrix} $$
        Rodando essa função no programa, temos os resultados

        \begin{table}[h]
            \centering
            \begin{tabular}{r|rrrrrrrr}
                Paraboloid & $\Vert\nabla f_*\Vert$ & it & $f$ & $\nabla f $ & $\nabla^2f$ & armijo & norma & ângulo \\
                \hline
                gradquad & 0.63E-05  &       7  &      36   &      8    &     0    &    22    &     0    &     0 \\
                gradcube & 0.99E-05  &    5926  &   11907   &   5927    &     0    &    55    &     0    &     0 \\
                bfgsquad & 0.19E-10  &       2  &       9   &      3    &     0    &     5    &     0    &     0 \\
                bfgscube & 0.16E-07  &       7  &      18   &      8    &     0    &     4    &     0    &     0 \\
                newtquad & 0.57E-13  &       1  &       2   &      2    &     1    &     0    &     0    &     0 \\
                newtcube & 0.57E-13  &       1  &       2   &      2    &     1    &     0    &     0    &     0 \\
            \end{tabular}
            \caption{Saída do programa para o parabolóide, com $\epsilon = \theta = 10^{-5} $, $\gamma = \sigma = 10^{-4}$.}
        \end{table}
        em que fica claro que todos os métodos convergiram para um ponto estacionário (dada a precisão exigida).

        Mas é interessante notar que o método Newton converge em apenas uma iteração, enquanto que os outros não.

        Há diversos fatores que levam a isso, mas os principais são:
        \begin{enumerate}
            \item A função é quadrática, ou seja, a aproximação que Newton faz utilizando a hessiana acaba sendo a própria função, e então o método é capaz de escolher a direção que de fato aponta para a solução, permitindo a convergência instantânea.
            \item O BFGS fica logo atrás, mas como inicialmente ele ainda não tem informação suficiente para aproximar a hessiana, ele não consegue atingir diretamente a solução. Entretanto, ele não demora muito para tal. Nota-se também que a interpolação quadrática é mais eficiente, justamente pela função também o ser.
            \item O método do gradiente, entretanto, apesar de ter uma boa eficiência com a interpolação quadrática (que faz grande parte do trabalho em minimizar a função), não se dá muito bem com a interpolação cúbica. Isso se dá em grande parte devido à diferença de escalas do problema (o fator 1000 em $x_2$), com a qual esse algorítmo não lida muito bem. E, uma vez removida a contribuição da interpolação, isso se torna bem evidente.
        \end{enumerate}

        Para mostrar mais claramente a contribuição da diferença de escala para o problema, tomando o parabolóide regular:

        \begin{table}[h!]
            \centering
            \begin{tabular}{r|rrrrrrrr}
                Regular Paraboloid & $\Vert\nabla f_*\Vert$ & it & $f$ & $\nabla f $ & $\nabla^2f$ & armijo & norma & ângulo \\
                \hline
                gradquad&  0.00E+00&         1&         3&         2&         0&         1&         0&         0 \\
                gradcube&  0.00E+00&         1&         3&         2&         0&         1&         0&         0 \\
                bfgsquad&  0.00E+00&         1&         3&         2&         0&         1&         0&         0 \\
                bfgscube&  0.00E+00&         1&         3&         2&         0&         1&         0&         0 \\
                newtquad&  0.80E-13&         1&         2&         2&         1&         0&         0&         0 \\
                newtcube&  0.80E-13&         1&         2&         2&         1&         0&         0&         0 \\
            \end{tabular}
            \caption{Saída do programa para o parabolóide regular, com $\epsilon = \theta = 10^{-5} $, $\gamma = \sigma = 10^{-4}$.}
        \end{table}
        ou seja, a interpolação sozinha é capaz de minimizar a função, pois a direção de busca inicial de todos já aponta para a solução.

        \newpage
        Outro exemplo interessante é o seguinte:

        \begin{table}[h!]
            \centering
            \begin{tabular}{r|rrrrrrrr}
                4 Powell Badly Scaled & $\Vert\nabla f_*\Vert$ & it & $f$ & $\nabla f $ & $\nabla^2f$ & armijo & norma & ângulo \\
                \hline
                gradquad&        FC \\
                gradcube&        FC \\
                bfgsquad&  0.60E-05&       122&       530&        93&         0&       346&         6&        30 \\
                bfgscube&  0.91E-05&       309&      1412&       229&         0&       956&        15&        81 \\
                newtquad&  0.94E-05&       297&       979&       298&       297&       385&       169&       452 \\
                newtcube&  0.94E-05&       297&       979&       298&       297&       385&       169&       452 \\
            \end{tabular}
            \caption{Saída do programa para o problema 4 de Moré-Garbow-Hillstrom (\emph{Powell Badly Scaled}) , com $\epsilon = \theta = 10^{-5} $, $\gamma = \sigma = 10^{-4}$.}
        \end{table}
        cuja análise não é tão simples, mas o interessante é que o método do gradiente não convergiu (FC, \emph{failure to converge}, indica que o algorítimo não conseguiu convergir em menos de 1.000.000 avaliações da função objetiva).

        Novamente, como o nome sugere, isso se dá pelas diferenças de escala do problema.
        Vale também notar que o método de Newton apresenta grandes problemas com a condição do ângulo, pois o melhor que podemos fazer para corrigir isso é criar uma aproximação da hessiana que ``tende'' à identidade (que nos daria a direção de máxima descida), enquanto que o BFGS tem a liberdade de a tomar imediatamente.

        \subsection*{Convergência}
            Analisando agora a função de Rosenbrock:
                $$ f(x) = (1 - x_1)^2 + 100(x_2 - x_1^2)^2 $$
            com os resultados

            \begin{table}[h!]
                \centering
                \begin{tabular}{r|rrrrrrrr}
                    Rosenbrock & $\Vert\nabla f_*\Vert$ & it & $f$ & $\nabla f $ & $\nabla^2f$ & armijo & norma & ângulo \\
                    \hline
                    gradquad&  0.99E-05&     10668&     21373&     10669&         0&        37&         0&         0 \\
                    gradcube&  0.99E-05&     10526&     21093&     10527&         0&        41&         0&         0 \\
                    bfgsquad&  0.79E-05&       103&       299&        76&         0&       149&         0&        28 \\
                    bfgscube&  0.67E-05&        32&       151&        27&         0&        99&         0&         6 \\
                    newtquad&  0.40E-05&        11&        23&        12&        11&         1&         0&        13 \\
                    newtcube&  0.40E-05&        11&        23&        12&        11&         1&         0&        13 \\
                \end{tabular}
                \caption{Saída do programa para a função de Rosenbrock , com $\epsilon = \theta = 10^{-5} $, $\gamma = \sigma = 10^{-4}$.}
            \end{table}
            e cujos 11 últimos valores\footnote{No caso, para o método de Newton, são todos os valores da norma do gradiente, já que ele teve somente 11 iterações.} de $\Vert\nabla f(x)\Vert$ para todos os algorítmos com interpolação quadrática são:
            \begin{table}[h]
                \centering
                \begin{tabular}{ccc}
                    Gradiente & BFGS & Newton \\
                    \hline
                    0.0000120 & 0.0107588 & 447.2135955 \\
                    0.0000280 & 0.0094878 & 176.8260698 \\
                    0.0000103 & 0.0094878 & 15.9468161 \\
                    0.0000168 & 0.0003073 & 3.0784906 \\
                    0.0000230 & 0.0001685 & 5.2304703 \\
                    0.0000123 & 0.0027361 & 4.5961242 \\
                    0.0000287 & 0.0015423 & 1.9377240 \\
                    0.0000105 & 0.0015423 & 1.7015645 \\
                    0.0000214 & 0.0000150 & 0.3193714 \\
                    0.0000146 & 0.0003777 & 0.1137346 \\
                    0.0000271 & 0.0003777 & 0.0017722 \\
                    0.0000099 & 0.0000079 & 0.0000040
                \end{tabular}
            \end{table}

            \newpage
            podemos observar algumas coisas:
            \begin{enumerate}
                \item O método do gradiente não avança muito, apenas flutua um pouco o valor, eventualmente chegando na precisão exigida. Mas, como demonstrado pelo seu número de iterações, demora muito.
                \item O BFGS já é melhor, começando com um valor três ordens de grandeza maior e ainda assim atingindo uma precisão melhor no mesmo número de iterações. Isso sem contar com os valores duplicados, que são devido à condição do ângulo, a qual força o algoritmo a repetir a iteração com a matriz identidade. Mesmo assim, os valores variam bastante, então é difícil dizer qual sua taxa de convergência.
                \item Já o método de Newton apresenta de longe a melhor convergência, ainda mais nas últimas três iterações, nas quais a ordem de grandeza dos valores cai quase quadraticamente ($10^{-1}$, $10^{-3}$, $10^{-5}$), que era o esperado.
            \end{enumerate}

            Entretanto, nem sempre se observam essas diferenças, como no seguinte problema:

            \begin{table}[h!]
                \centering
                \begin{tabular}{r|rrrrrrrr}
                    8 Penalty I & $\Vert\nabla f_*\Vert$ & it & $f$ & $\nabla f $ & $\nabla^2f$ & armijo & norma & ângulo \\
                    \hline
                    gradquad&  0.89E-05&         7&        26&         8&         0&        12&         0&         0 \\
                    gradcube&  0.90E-05&         8&        33&         9&         0&        17&         0&         0 \\
                    bfgsquad&  0.90E-05&        10&        22&        11&         0&         2&         0&         0 \\
                    bfgscube&  0.90E-05&        10&        22&        11&         0&         2&         0&         0 \\
                    newtquad&  0.17E-05&        33&        78&        34&        33&        12&         0&         0 \\
                    newtcube&  0.17E-05&        33&        78&        34&        33&        12&         0&         0 \\
                \end{tabular}
                \caption{Saída do programa para o problema 11 de Moré-Garbow-Hillstrom (\emph{Penalty I}), com $\epsilon = \theta = 10^{-5} $, $\gamma = \sigma = 10^{-4}$.}
            \end{table}
            no qual, curiosamente, a ordem dos métodos por número de iterações é o contrário do que se esperaria!

            Além disso, houve também casos patológicos:

            \begin{table}[h!]
                \centering
                \begin{tabular}{r|r}
                    11 Brown and Dennis & $\Vert\nabla f_*\Vert$ \\
                    \hline
                    gradquad&        FC \\
                    gradcube&        FC \\
                    bfgsquad&       NaN \\
                    bfgscube&       NaN \\
                    newtquad&        FC \\
                    newtcube&        FC \\
                \end{tabular}
                \caption{Saída do programa para o problema 11 de Moré-Garbow-Hillstrom (\emph{Brown and Dennis}), com $\epsilon = \theta = 10^{-5} $, $\gamma = \sigma = 10^{-4}$.}
            \end{table}
            em que \texttt{NaN} indica que o método convergiu para um ponto que não era estacionário (e portanto aparece um \texttt{NaN} durante o cálculo de $H$). No entanto, não foi econtrada nenhuma referência que explicasse os detalhes dessa função, nem o que levaria o método a isso.

    \section*{Conclusão}
        Há uma grande liberdade em se adotar estratégias dentro do paradigma de busca linear, e no geral é necessário balancear simplicidade e eficiência. Mesmo assim, isso varia muito com o problema em questão, o que impossibilita a escolha do ``melhor'' método, já que um método supreendentemente pode se mostrar o mais eficaz naquele contexto.

        Mas no geral há métodos que se mostram eficazes em boa parte dos problemas e acabam se tornando padrões, como o BFGS, que evita os problemas do método do gradiente e tem a vantagem de não exigir informações de segunda ordem do problema, como o de Newton.

        Portanto, a escolha de um algorítmo às vezes se mostra por tentativa e erro, a não ser que o problema em questão tenha características bem conhecidas que possam ser exploradas por algum método particular.
        
    \bibliography{EP1}
\end{document}
