%%% LaTeX Template
%%% This template is made for project reports
%%%	You may adjust it to your own needs/purposes
%%%
%%% Copyright: http://www.howtotex.com/
%%% Date: March 2011

%%% Preamble
\documentclass[paper=a4, fontsize=11pt]{scrartcl}	% Article class of KOMA-script with 11pt font and a4 format
\usepackage[utf8]{inputenc}
\usepackage[T1]{fontenc}
\usepackage{fourier}

\usepackage[brazilian]{babel}										% Brazilian portuguese language/hyphenation
\usepackage[protrusion=true,expansion=true]{microtype}				% Better typography
\usepackage{amsmath,amsfonts,amsthm}								% Math packages
\usepackage[pdftex]{graphicx}										% Enable pdflatex
\usepackage{url}


%%% Custom sectioning (sectsty package)
\usepackage{sectsty}												% Custom sectioning (see below)
\allsectionsfont{\centering \normalfont\scshape}	% Change font of al section commands


%%% Custom headers/footers (fancyhdr package)
\usepackage{fancyhdr}
\pagestyle{fancyplain}
\fancyhead{}														% No page header
\fancyfoot[C]{}													% Empty
\fancyfoot[R]{\thepage}									% Pagenumbering
\renewcommand{\headrulewidth}{0pt}			% Remove header underlines
\renewcommand{\footrulewidth}{0pt}				% Remove footer underlines
\setlength{\headheight}{13.6pt}


%%% Equation and float numbering
\numberwithin{equation}{section}		% Equationnumbering: section.eq#
\numberwithin{figure}{section}			% Figurenumbering: section.fig#
\numberwithin{table}{section}				% Tablenumbering: section.tab#


%%% Maketitle metadata
\newcommand{\horrule}[1]{\rule{\linewidth}{#1}} 	% Horizontal rule

\title{
		%\vspace{-1in}
		\usefont{OT1}{bch}{b}{n}
		\normalfont \normalsize \textsc{Programação Não-Linear} \\ [25pt]
		\huge Minimização com Restrições de Igualdade \\
		\LARGE Método de penalidade
}


\author{
		\normalfont 								\normalsize
        Lucas Magno\\[-3pt]		\normalsize
}
\date{}


%%% Begin document
\begin{document}
\maketitle
\begin{figure}
	\centering
	% GNUPLOT: LaTeX picture
\setlength{\unitlength}{0.240900pt}
\ifx\plotpoint\undefined\newsavebox{\plotpoint}\fi
\sbox{\plotpoint}{\rule[-0.200pt]{0.400pt}{0.400pt}}%
\begin{picture}(1500,900)(0,0)
\sbox{\plotpoint}{\rule[-0.200pt]{0.400pt}{0.400pt}}%
\put(431.0,131.0){\rule[-0.200pt]{4.818pt}{0.400pt}}
\put(411,131){\makebox(0,0)[r]{$0$}}
\put(1139.0,131.0){\rule[-0.200pt]{4.818pt}{0.400pt}}
\put(431.0,313.0){\rule[-0.200pt]{4.818pt}{0.400pt}}
\put(411,313){\makebox(0,0)[r]{$0.5$}}
\put(1139.0,313.0){\rule[-0.200pt]{4.818pt}{0.400pt}}
\put(431.0,495.0){\rule[-0.200pt]{4.818pt}{0.400pt}}
\put(411,495){\makebox(0,0)[r]{$1$}}
\put(1139.0,495.0){\rule[-0.200pt]{4.818pt}{0.400pt}}
\put(431.0,677.0){\rule[-0.200pt]{4.818pt}{0.400pt}}
\put(411,677){\makebox(0,0)[r]{$1.5$}}
\put(1139.0,677.0){\rule[-0.200pt]{4.818pt}{0.400pt}}
\put(431.0,859.0){\rule[-0.200pt]{4.818pt}{0.400pt}}
\put(411,859){\makebox(0,0)[r]{$2$}}
\put(1139.0,859.0){\rule[-0.200pt]{4.818pt}{0.400pt}}
\put(431.0,131.0){\rule[-0.200pt]{0.400pt}{4.818pt}}
\put(431,90){\makebox(0,0){$0$}}
\put(431.0,839.0){\rule[-0.200pt]{0.400pt}{4.818pt}}
\put(613.0,131.0){\rule[-0.200pt]{0.400pt}{4.818pt}}
\put(613,90){\makebox(0,0){$0.5$}}
\put(613.0,839.0){\rule[-0.200pt]{0.400pt}{4.818pt}}
\put(795.0,131.0){\rule[-0.200pt]{0.400pt}{4.818pt}}
\put(795,90){\makebox(0,0){$1$}}
\put(795.0,839.0){\rule[-0.200pt]{0.400pt}{4.818pt}}
\put(977.0,131.0){\rule[-0.200pt]{0.400pt}{4.818pt}}
\put(977,90){\makebox(0,0){$1.5$}}
\put(977.0,839.0){\rule[-0.200pt]{0.400pt}{4.818pt}}
\put(1159.0,131.0){\rule[-0.200pt]{0.400pt}{4.818pt}}
\put(1159,90){\makebox(0,0){$2$}}
\put(1159.0,839.0){\rule[-0.200pt]{0.400pt}{4.818pt}}
\put(431.0,131.0){\rule[-0.200pt]{0.400pt}{175.375pt}}
\put(431.0,131.0){\rule[-0.200pt]{175.375pt}{0.400pt}}
\put(1159.0,131.0){\rule[-0.200pt]{0.400pt}{175.375pt}}
\put(431.0,859.0){\rule[-0.200pt]{175.375pt}{0.400pt}}
\put(310,495){\makebox(0,0){$x_2$}}
\put(795,29){\makebox(0,0){$x_1$}}
\put(999,818){\makebox(0,0)[r]{"teste1.dat"}}
\put(795,495){\makebox(0,0){$\bullet$}}
\put(448,148){\makebox(0,0){$\bullet$}}
\put(461,161){\makebox(0,0){$\bullet$}}
\put(483,183){\makebox(0,0){$\bullet$}}
\put(512,212){\makebox(0,0){$\bullet$}}
\put(543,243){\makebox(0,0){$\bullet$}}
\put(570,270){\makebox(0,0){$\bullet$}}
\put(588,288){\makebox(0,0){$\bullet$}}
\put(600,300){\makebox(0,0){$\bullet$}}
\put(606,306){\makebox(0,0){$\bullet$}}
\put(610,310){\makebox(0,0){$\bullet$}}
\put(611,311){\makebox(0,0){$\bullet$}}
\put(612,312){\makebox(0,0){$\bullet$}}
\put(613,313){\makebox(0,0){$\bullet$}}
\put(613,313){\makebox(0,0){$\bullet$}}
\put(613,313){\makebox(0,0){$\bullet$}}
\put(613,313){\makebox(0,0){$\bullet$}}
\put(613,313){\makebox(0,0){$\bullet$}}
\put(613,313){\makebox(0,0){$\bullet$}}
\put(613,313){\makebox(0,0){$\bullet$}}
\put(613,313){\makebox(0,0){$\bullet$}}
\put(613,313){\makebox(0,0){$\bullet$}}
\put(613,313){\makebox(0,0){$\bullet$}}
\put(613,313){\makebox(0,0){$\bullet$}}
\put(613,313){\makebox(0,0){$\bullet$}}
\put(613,313){\makebox(0,0){$\bullet$}}
\put(1069,818){\makebox(0,0){$\bullet$}}
\put(999,777){\makebox(0,0)[r]{c1(x) = 0}}
\multiput(1019,777)(20.756,0.000){5}{\usebox{\plotpoint}}
\put(1119,777){\usebox{\plotpoint}}
\put(431,495){\usebox{\plotpoint}}
\put(431.00,495.00){\usebox{\plotpoint}}
\put(445.68,480.32){\usebox{\plotpoint}}
\put(460.35,465.65){\usebox{\plotpoint}}
\put(475.03,450.97){\usebox{\plotpoint}}
\put(489.71,436.29){\usebox{\plotpoint}}
\put(504.38,421.62){\usebox{\plotpoint}}
\put(519.06,406.94){\usebox{\plotpoint}}
\put(533.73,392.27){\usebox{\plotpoint}}
\put(548.41,377.59){\usebox{\plotpoint}}
\put(563.09,362.91){\usebox{\plotpoint}}
\put(577.76,348.24){\usebox{\plotpoint}}
\put(592.44,333.56){\usebox{\plotpoint}}
\put(607.12,318.88){\usebox{\plotpoint}}
\put(621.79,304.21){\usebox{\plotpoint}}
\put(636.47,289.53){\usebox{\plotpoint}}
\put(651.15,274.85){\usebox{\plotpoint}}
\put(665.82,260.18){\usebox{\plotpoint}}
\put(680.50,245.50){\usebox{\plotpoint}}
\put(695.17,230.83){\usebox{\plotpoint}}
\put(709.85,216.15){\usebox{\plotpoint}}
\put(724.53,201.47){\usebox{\plotpoint}}
\put(739.20,186.80){\usebox{\plotpoint}}
\put(753.88,172.12){\usebox{\plotpoint}}
\put(768.56,157.44){\usebox{\plotpoint}}
\put(783.23,142.77){\usebox{\plotpoint}}
\put(795,131){\usebox{\plotpoint}}
\put(431.0,131.0){\rule[-0.200pt]{0.400pt}{175.375pt}}
\put(431.0,131.0){\rule[-0.200pt]{175.375pt}{0.400pt}}
\put(1159.0,131.0){\rule[-0.200pt]{0.400pt}{175.375pt}}
\put(431.0,859.0){\rule[-0.200pt]{175.375pt}{0.400pt}}
\end{picture}

\end{figure}

\begin{figure}
	\centering
	\input{teste3}
\end{figure}

\begin{figure}
	\centering
	\input{teste4}
\end{figure}
%%% End document
\end{document}
