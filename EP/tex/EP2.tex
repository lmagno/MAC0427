\documentclass[a4paper,11pt]{article}
\usepackage[T1]{fontenc}
\usepackage[utf8]{inputenc}
\usepackage{graphicx}
\usepackage{amsmath}
\usepackage[left=2.5cm,right=2.5cm,top=2.0cm,bottom=1.5cm]{geometry}
\usepackage[hidelinks]{hyperref}
\usepackage{indentfirst}
\usepackage{caption}
\usepackage{subcaption}
\usepackage{algorithm}
\usepackage{algpseudocode}
\usepackage{txfonts} % Para usar o símbolo dos números reais (\varmathbb{R})
\usepackage[brazilian]{babel}
% \usepackage{natbib}
% \bibliographystyle{plainnat}
\usepackage{enumitem}
\usepackage{dirtree}
\usepackage{import}
\usepackage{epstopdf}
\usepackage[subfolder]{gnuplottex}
\setlist[description]{leftmargin=2\parindent,labelindent=2\parindent}

\date{}
\author{Lucas Magno \\ 7994983}
\title{Programação Não-Linear \\ Algoritmos de Busca Linear}

\begin{document}

    \begin{figure}
        \centering
        \begin{gnuplot}[terminal=epslatex,terminaloptions=]
            min = -1
            max = 2
            set xrange [min:max]
            set yrange [min:max]

            unset border

            set xzeroaxis lt -1 lw 3
            set xtics axis (1)
            set label "$x$" at max-0.1,-.1

            set yzeroaxis lt -1 lw 3
            set ytics axis (1)
            set label "$y$" at -.1,max-0.1


            set label "$(\\frac{1}{2},\\frac{1}{2})$" at 0.55,0.55 left
            set size square
            plot "../data/teste1.dat" pt 7 ps 0.5 title "", \
                 1-x w lines lw 3 title "$x + y = 1$"
        \end{gnuplot}
    \end{figure}

    \begin{figure}
        \centering
        \begin{gnuplot}[terminal=epslatex,terminaloptions=]
            xmin = -1
            xmax = 3
            ymin = -2
            ymax = 2
            set xrange [xmin:xmax]
            set yrange [ymin:ymax]

            set key tmargin
            unset border

            set cntrparam levels discrete 0
            set view map
            unset surface
            set isosamples 1000,1000
            set contour

            c1(x, y) = (x-1)**2 + (y-1)**2 - 1
            c2(x, y) = (x-1)**2 + (y+1)**2 - 1

            set table 'gnuplottex/c1_2'
            splot c1(x, y)
            unset table

            set table 'gnuplottex/c2_2'
            splot c2(x, y)
            unset table

            set xzeroaxis lt -1 lw 3
            set xtics axis (1)
            set label "$x$" at xmax-0.1,-.1

            set yzeroaxis lt -1 lw 3
            unset ytics
            set label "$y$" at -.1,ymax-0.1 right


            set size square
            plot "../data/teste2.dat" pt 1 ps 1 t "", \
                 "../data/c1_2" u 1:2 title "$(x-1)^2+(y-1)^2 = 1$" w lines lw 3,\
                 "../data/c2_2" u 1:2 title "$(x-1)^2+(y+1)^2 = 1$" w lines lw 3
        \end{gnuplot}
    \end{figure}
\end{document}
