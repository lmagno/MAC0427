Uma qualificação possível (de fato a mais fraca \citep{Wachsmuth201378}) é a
\emph{Linear Independence Constraint Qualification} ou LICQ, definida a seguir.

\begin{definition}
    Um ponto $x^*$ é dito satisfazer LICQ se e somente o conjunto formado pelos
    gradientes das restrições em $x^*$
    $$ \{ \nabla c_1(x^*), \dots, \nabla c_m(x^*)\} $$
    é linearmente independente.
\end{definition}

Assim, podemos agora enunciar a condição necessária de primeira ordem:

\begin{theorem}
    Seja $x^*$ um minimizador local do problema \ref{eq:problema} que satisfaça LICQ,
    então existe $\lambda^* \in \varmathbb{R}^m$ tal que

    \begin{equation}
        \nabla f(x^*) = \sum_{i = 0}^{m} \lambda_i^* \nabla c_i(x^*)
    \end{equation}

    onde $\lambda^*$ é conhecido como o vetor de multiplicadores de langrange.
\end{theorem}

Uma vez que temos uma condição de otimalidade, podemos a utilizar como critério
de parada no algoritmo a ser implementado.
