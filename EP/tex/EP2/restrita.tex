Diferentemente da otimização irrestrita, neste trabalho consideraremos problemas de
otimização nos quais a região viável não é todo o $\varmathbb{R}^n$ e sim um subconjunto
seu determinado por várias equações não-lineares.

Podemos formular o problema na seguinte forma:
\begin{equation}
    \label{eq:problema}
    \begin{aligned}
        & \text{minimizar} & & f(x) \\
        & \text{sujeita a} & & c_i(x) = 0, \quad i = 1, m
    \end{aligned}
\end{equation}
onde $x  \in \varmathbb{R}^n $ e \footnote{Assumindo $f$ e $c_i$ continuamente diferenciáveis.}
\begin{equation*}
    \begin{aligned}
        & f: & \varmathbb{R}^n \rightarrow \varmathbb{R} & &  \\
        & c_i: & \varmathbb{R}^n \rightarrow \varmathbb{R} & & ,\quad  i = 1, m
    \end{aligned}
\end{equation*}

Portanto, não podemos mais utilizar os algoritmos já desenvolvidos e precisamos
definir as condições de otimalidade neste novo problema. Para isto, é preciso
que se defina uma condição de qualificação, isto é, hipóteses adicionais que nos
permitirão qualificar os minimizadores.

\newpage
\subsection{LICQ e Multiplicadores de Lagrange}
    \subimport{restrita/}{lagrange}
