Mas como lidar com as restrições? A solução simples é não lidar com elas! \citep{Friedlander94}
Quer dizer, podemos definir o seguinte problema irrestrito

\begin{equation}
    \label{eq:subproblema}
    \underset{x \in \varmathbb{R}^n}{\text{minimizar}} \quad Q(x, \mu) = f(x) + \frac{1}{2\mu}\sum_{i = 0}^m c_i(x)^2
\end{equation}
para dado $\mu \in \varmathbb{R}$, no qual pontos fora da região viável recebem uma \emph{penalidade} com peso
$1/2\mu$ e assim, para $\mu$ suficientemente grande, espera-se que o algoritmo
descarte pontos fora da região viável.

De fato, pode-se mostrar que, sendo a sequência $\{ \mu_k\}_{k=0}^\infty$ tal que

\begin{equation}
    \mu_k \rightarrow 0 \quad (k \rightarrow \infty)
\end{equation}
então a sequência $\{x_k\}_{k = 0}^\infty$ de soluções dos subproblemas
\begin{equation*}
    \underset{x \in \varmathbb{R}^n}{\text{minimizar}} \quad Q(x, \mu_k)
\end{equation*}
tende a $x^*$, que é solução do problema \ref{eq:problema}.

No entanto, isso requer a solução exata do subproblema \ref{eq:subproblema}.
Por outro lado, se tivermos $x_k$ tal que
\begin{equation*}
    \| \nabla Q(x_k, \mu_k)\| \leq r_k, \quad r_k \in \varmathbb{R}
\end{equation*}
e
$$ r_k \rightarrow 0$$
também é possível mostrar que $x_k \rightarrow x^*$ e
$$ \lambda_i^k = -\frac{c_i(x_k)}{\mu_k} \rightarrow \lambda_i^* \quad i = 1, m$$
o que é muito mais útil
de um ponto de vista prático.
