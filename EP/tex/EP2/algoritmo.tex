Tendo isto, podemos montar um algoritmo a fim de resolver o problema \ref{eq:problema}.

Dados $\{\mu_k\}_{k = 0}^\infty$ e $\{r_k\}_{k = 0}^\infty$ como definidos acima
e $x_0^S \in \varmathbb{R}^n$
\begin{algorithm}[h]
    \caption{Método de Penalização}
    \label{alg:penalizacao}
    \begin{algorithmic}
        \For {$k = 0,1,2, \dots$}
            \State \textbf{Passo 1} Usando $x_k^S$ como ponto inicial, encontrar $x_k$ solução
                   aproximada do subproblema \ref{eq:subproblema}
                   tal que $\| \nabla Q(x_k, \mu_k) \| \leq r_k$

            \State \textbf{Passo 2} Se $x_k$ satisfaz o critério \ref{eq:parada}, parar.
            \State \textbf{Passo 3} Escolher $x_{k+1}^S$
            \State \textbf{Passo 4} Escolher $\mu_{k+1}$
        \EndFor
    \end{algorithmic}
\end{algorithm}

Por motivo de simplicidade, serão usadas as fórmulas
$$ r_k = \epsilon$$
$$ x_{k+1}^S \gets x_k$$
$$ \mu_{k+1} \gets \mu_k/2$$
onde $\epsilon$ é a tolerância definida para o programa.

\subsection{Mal condicionamento}
    Como método no primeiro passo foi escolhido o de Newton, no entanto é necessário
    uma observação. Para encontrar uma direção de descida $d$, o método resolve um sistema
    na forma
    $$ \nabla^2 Q(x, \mu) d = -\nabla Q(x, \mu) $$

    Porém, como
    \begin{equation}
        \begin{aligned}
            \nabla^2 Q(x, \mu) &= \nabla^2 f(x) + \frac{1}{\mu}\sum_{i = 0}^m c_i(x)\nabla^2 c_i(x) + \frac{1}{\mu}\sum_{i = 0}^m \nabla c_i(x)\nabla c_i(x)^\top \\
                               &= \nabla^2 \mathcal{L}(x) + \frac{1}{\mu}\sum_{i = 0}^m \nabla c_i(x)\nabla c_i(x)^\top
        \end{aligned}
    \end{equation}
    nota-se que o último termo é uma matriz de posto $m$ e, para $\mu$ suficientemente
    pequeno, vale que $\nabla^2 Q(x, \mu)$ é mal condicionada, o que pode inviabilizar
    a solução do sistema.

    Em seu lugar, podemos utilizar um sistema equivalente introduzindo uma variável
    auxiliar $z \in \varmathbb{R}^m$:

    \begin{equation*}
        \begin{bmatrix}
            \nabla^2 \mathcal{L}(x) & A(x)^\top \\
            A(x) & -\mu I
        \end{bmatrix}
        \begin{bmatrix}
            d \\
            z
        \end{bmatrix} \
        = \
        \begin{bmatrix}
            -\nabla Q(x, \mu) \\
            0
        \end{bmatrix}
    \end{equation*}
    onde
    \begin{equation*}
        A(x) = \
        \begin{bmatrix}
            \nabla c_1(x) \\
            \vdots \\
            \nabla c_m(x)
        \end{bmatrix}
    \end{equation*}

    Esse sistema não apresenta mal condicionamento devido a
    $\mu$ pequeno e portanto será utilizado na implementação do método de Newton,
    mas fora isso o método se mantém o mesmo do trabalho anterior.
