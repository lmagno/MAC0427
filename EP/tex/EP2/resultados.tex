Foram escolhidos quatro problemas para testar o programa, definidos no $\varmathbb{R}^2$
a fim de ilustração, cujos resultados foram:

\begin{table}[h!]
    \centering
	\label{tab:resultados}
    \begin{tabular}{r|rrrrrrrrr}
         & $\Vert c^*\Vert$ & $\Vert\nabla \mathcal{L}^*\Vert$ & sub & $f$ & $\nabla f $ & $\nabla^2f$ & armijo & norma & ângulo \\
        \hline
        \subimport{../../aux/}{output.dat}
    \end{tabular}
    \caption{Saída do programa para os quatro problemas testados, com
	$\epsilon = 10^{-6}$, $\theta = 10^{-5} $, $\gamma = \sigma = 10^{-4}$ e
	ponto inicial $(1, 1)$.}
\end{table}

E a seguir vamos analisar cada um.

\subsection{Não-Linearidade das restrições}
    Os problemas 1 e 2 são, respectivamente

    \noindent\begin{minipage}{.5\linewidth}
        \begin{equation*}
			\begin{aligned}
				& \text{minimizar} & & x^2 + y^2 \\
				& \text{sujeita a} & & x + y - 1 = 0
			\end{aligned}
        \end{equation*}
    \end{minipage}%
    \begin{minipage}{.5\linewidth}
        \begin{equation*}
			\begin{aligned}
				& \text{minimizar} & & x^2 + y^2 \\
				& \text{sujeita a} & & (x + y - 1)^3 = 0
			\end{aligned}
        \end{equation*}
    \end{minipage}

	Ou seja, ambos consistem em achar o ponto da reta $x + y = 1$ mais próximo
	à origem e a única diferença entre eles é que a restrição é linear no primeiro,
	mas não no segundo.

	Visualizando o progresso do algoritmo através dos gráficos

	\noindent\begin{figure}[h!]
		\noindent\begin{minipage}[l]{.45\linewidth}
			\scalebox{.7}{% GNUPLOT: LaTeX picture
\setlength{\unitlength}{0.240900pt}
\ifx\plotpoint\undefined\newsavebox{\plotpoint}\fi
\sbox{\plotpoint}{\rule[-0.200pt]{0.400pt}{0.400pt}}%
\begin{picture}(1500,900)(0,0)
\sbox{\plotpoint}{\rule[-0.200pt]{0.400pt}{0.400pt}}%
\put(431.0,131.0){\rule[-0.200pt]{4.818pt}{0.400pt}}
\put(411,131){\makebox(0,0)[r]{$0$}}
\put(1139.0,131.0){\rule[-0.200pt]{4.818pt}{0.400pt}}
\put(431.0,313.0){\rule[-0.200pt]{4.818pt}{0.400pt}}
\put(411,313){\makebox(0,0)[r]{$0.5$}}
\put(1139.0,313.0){\rule[-0.200pt]{4.818pt}{0.400pt}}
\put(431.0,495.0){\rule[-0.200pt]{4.818pt}{0.400pt}}
\put(411,495){\makebox(0,0)[r]{$1$}}
\put(1139.0,495.0){\rule[-0.200pt]{4.818pt}{0.400pt}}
\put(431.0,677.0){\rule[-0.200pt]{4.818pt}{0.400pt}}
\put(411,677){\makebox(0,0)[r]{$1.5$}}
\put(1139.0,677.0){\rule[-0.200pt]{4.818pt}{0.400pt}}
\put(431.0,859.0){\rule[-0.200pt]{4.818pt}{0.400pt}}
\put(411,859){\makebox(0,0)[r]{$2$}}
\put(1139.0,859.0){\rule[-0.200pt]{4.818pt}{0.400pt}}
\put(431.0,131.0){\rule[-0.200pt]{0.400pt}{4.818pt}}
\put(431,90){\makebox(0,0){$0$}}
\put(431.0,839.0){\rule[-0.200pt]{0.400pt}{4.818pt}}
\put(613.0,131.0){\rule[-0.200pt]{0.400pt}{4.818pt}}
\put(613,90){\makebox(0,0){$0.5$}}
\put(613.0,839.0){\rule[-0.200pt]{0.400pt}{4.818pt}}
\put(795.0,131.0){\rule[-0.200pt]{0.400pt}{4.818pt}}
\put(795,90){\makebox(0,0){$1$}}
\put(795.0,839.0){\rule[-0.200pt]{0.400pt}{4.818pt}}
\put(977.0,131.0){\rule[-0.200pt]{0.400pt}{4.818pt}}
\put(977,90){\makebox(0,0){$1.5$}}
\put(977.0,839.0){\rule[-0.200pt]{0.400pt}{4.818pt}}
\put(1159.0,131.0){\rule[-0.200pt]{0.400pt}{4.818pt}}
\put(1159,90){\makebox(0,0){$2$}}
\put(1159.0,839.0){\rule[-0.200pt]{0.400pt}{4.818pt}}
\put(431.0,131.0){\rule[-0.200pt]{0.400pt}{175.375pt}}
\put(431.0,131.0){\rule[-0.200pt]{175.375pt}{0.400pt}}
\put(1159.0,131.0){\rule[-0.200pt]{0.400pt}{175.375pt}}
\put(431.0,859.0){\rule[-0.200pt]{175.375pt}{0.400pt}}
\put(310,495){\makebox(0,0){$x_2$}}
\put(795,29){\makebox(0,0){$x_1$}}
\put(999,818){\makebox(0,0)[r]{"teste1.dat"}}
\put(795,495){\makebox(0,0){$\bullet$}}
\put(448,148){\makebox(0,0){$\bullet$}}
\put(461,161){\makebox(0,0){$\bullet$}}
\put(483,183){\makebox(0,0){$\bullet$}}
\put(512,212){\makebox(0,0){$\bullet$}}
\put(543,243){\makebox(0,0){$\bullet$}}
\put(570,270){\makebox(0,0){$\bullet$}}
\put(588,288){\makebox(0,0){$\bullet$}}
\put(600,300){\makebox(0,0){$\bullet$}}
\put(606,306){\makebox(0,0){$\bullet$}}
\put(610,310){\makebox(0,0){$\bullet$}}
\put(611,311){\makebox(0,0){$\bullet$}}
\put(612,312){\makebox(0,0){$\bullet$}}
\put(613,313){\makebox(0,0){$\bullet$}}
\put(613,313){\makebox(0,0){$\bullet$}}
\put(613,313){\makebox(0,0){$\bullet$}}
\put(613,313){\makebox(0,0){$\bullet$}}
\put(613,313){\makebox(0,0){$\bullet$}}
\put(613,313){\makebox(0,0){$\bullet$}}
\put(613,313){\makebox(0,0){$\bullet$}}
\put(613,313){\makebox(0,0){$\bullet$}}
\put(613,313){\makebox(0,0){$\bullet$}}
\put(613,313){\makebox(0,0){$\bullet$}}
\put(613,313){\makebox(0,0){$\bullet$}}
\put(613,313){\makebox(0,0){$\bullet$}}
\put(613,313){\makebox(0,0){$\bullet$}}
\put(1069,818){\makebox(0,0){$\bullet$}}
\put(999,777){\makebox(0,0)[r]{c1(x) = 0}}
\multiput(1019,777)(20.756,0.000){5}{\usebox{\plotpoint}}
\put(1119,777){\usebox{\plotpoint}}
\put(431,495){\usebox{\plotpoint}}
\put(431.00,495.00){\usebox{\plotpoint}}
\put(445.68,480.32){\usebox{\plotpoint}}
\put(460.35,465.65){\usebox{\plotpoint}}
\put(475.03,450.97){\usebox{\plotpoint}}
\put(489.71,436.29){\usebox{\plotpoint}}
\put(504.38,421.62){\usebox{\plotpoint}}
\put(519.06,406.94){\usebox{\plotpoint}}
\put(533.73,392.27){\usebox{\plotpoint}}
\put(548.41,377.59){\usebox{\plotpoint}}
\put(563.09,362.91){\usebox{\plotpoint}}
\put(577.76,348.24){\usebox{\plotpoint}}
\put(592.44,333.56){\usebox{\plotpoint}}
\put(607.12,318.88){\usebox{\plotpoint}}
\put(621.79,304.21){\usebox{\plotpoint}}
\put(636.47,289.53){\usebox{\plotpoint}}
\put(651.15,274.85){\usebox{\plotpoint}}
\put(665.82,260.18){\usebox{\plotpoint}}
\put(680.50,245.50){\usebox{\plotpoint}}
\put(695.17,230.83){\usebox{\plotpoint}}
\put(709.85,216.15){\usebox{\plotpoint}}
\put(724.53,201.47){\usebox{\plotpoint}}
\put(739.20,186.80){\usebox{\plotpoint}}
\put(753.88,172.12){\usebox{\plotpoint}}
\put(768.56,157.44){\usebox{\plotpoint}}
\put(783.23,142.77){\usebox{\plotpoint}}
\put(795,131){\usebox{\plotpoint}}
\put(431.0,131.0){\rule[-0.200pt]{0.400pt}{175.375pt}}
\put(431.0,131.0){\rule[-0.200pt]{175.375pt}{0.400pt}}
\put(1159.0,131.0){\rule[-0.200pt]{0.400pt}{175.375pt}}
\put(431.0,859.0){\rule[-0.200pt]{175.375pt}{0.400pt}}
\end{picture}
}
			\caption{Sequência de soluções $x_k$ para o problema 1.}
		\end{minipage}%
		\hspace{0.1\linewidth}%
		\begin{minipage}[l]{.45\linewidth}
			\scalebox{.7}{\input{teste2}}
			\caption{Sequência de soluções $x_k$ para o problema 2.}
		\end{minipage}
	\end{figure}

	fica claro que ambos chegam à mesma solução de forma similar. No entanto, tanto pelo gráfico
	quanto pela tabela de resultados \ref{tab:resultados}, nota-se que o segundo caso resolve um número
	maior de subproblemas e ainda assim termina com uma precisão pior do que o primeiro,
	então restrições não-lineares introduzem uma certa dificuldade comparadas às
	lineares para este algoritmo.

\subsection{LICQ}
	Para ilustrar a necessidade da LICQ para o método, dois outros problemas
	foram testados:

	\noindent\begin{minipage}{.5\linewidth}
		\begin{equation*}
			\begin{aligned}
				& \text{minimizar} & & x^2 + y^2 \\
				& \text{sujeita a} & & x = 1 \\
				&  & & y = 0
			\end{aligned}
		\end{equation*}
	\end{minipage}%
	\begin{minipage}{.5\linewidth}
		\begin{equation*}
			\begin{aligned}
				& \text{minimizar} & & x^2 + y^2 \\
				& \text{sujeita a} & & (x-1)^2+(y-1)^2 = 1 \\
				&  & & (x-1)^2+(y+1)^2 = 1
			\end{aligned}
		\end{equation*}
	\end{minipage}

	nos quais em ambos a região viável se restringe a um único ponto, $x^* = (1, 0)$,
	e cujos gradientes da função e das restrições nesse ponto são, respectivamente:

	\begin{equation*}
		\nabla f(x^*) =
		\begin{bmatrix}
			2 \\
			0
		\end{bmatrix},\quad
		\nabla c_1(x^*) =
		\begin{bmatrix}
			1 \\
			0
		\end{bmatrix},\quad
		\nabla c_2(x^*) =
		\begin{bmatrix}
			0 \\
			1
		\end{bmatrix}
	\end{equation*}

	\begin{equation*}
		\nabla f(x^*) =
		\begin{bmatrix}
			2 \\
			0
		\end{bmatrix},\quad
		\nabla c_1(x^*) =
		\begin{bmatrix}
			0 \\
			-2
		\end{bmatrix},\quad
		\nabla c_2(x^*) =
		\begin{bmatrix}
			0 \\
			2
		\end{bmatrix}
	\end{equation*}

	em que fica claro que, enquanto os gradientes das restrições do primeiro são
	linearmente independente, os do segundo não o são.

	Novamente, podemos observar os gráficos:

	\noindent\begin{figure}[h!]
		\noindent\begin{minipage}[l]{.45\linewidth}
			\scalebox{.7}{\input{teste3}}
			\caption{Sequência de soluções $x_k$ para o problema 3.}
		\end{minipage}%
		\hspace{0.1\linewidth}%
		\begin{minipage}[l]{.45\linewidth}
			\scalebox{.7}{\input{teste4}}
			\caption{Sequência de soluções $x_k$ para o problema 4.}
		\end{minipage}
	\end{figure}

	e se nota que ambos se aproximam da solução $x^*$. Apesar disso, como visto
	na tabela \ref{tab:resultados}, o programa não converge para o problema 4.

	A partir dos valores dos critérios de parada para os últimos pontos
	do algoritmo em ambos os problemas
	\begin{table}[h!]
		\centering
		\begin{minipage}{.45\linewidth}
			\centering
			\begin{tabular}{cc}
				$\| c(x) \|$ & $\| \nabla \mathcal{L}(x) \|$ \\
				\hline
				0.95E-05 &   0.54E-06 \\
				0.48E-05 &   0.11E-11 \\
				0.24E-05 &   0.11E-10 \\
				0.12E-05 &   0.28E-11 \\
				0.60E-06 &   0.91E-06 \\
			\end{tabular}
			\caption{Valores dos critérios de parada para a sequência $x_k$ do problema 3.}
		\end{minipage}%
		\hspace{0.1\linewidth}%
		\begin{minipage}[r]{.45\linewidth}
			\centering
			\begin{tabular}{cc}
				$\| c(x) \|$ & $\| \nabla \mathcal{L}(x) \|$ \\
				\hline
				0.10E-03 &   0.11E-07 \\
				0.63E-04 &   0.20E-06 \\
				0.40E-04 &   0.31E-06 \\
				0.25E-04 &   0.53E-06 \\
				0.16E-04 &   0.32E-07 \\
			\end{tabular}
			\caption{Valores dos critérios de parada para a sequência $x_k$ do problema 4.}
		\end{minipage}
	\end{table}

	somado ao fato de que o programa resolveu 26 subproblemas para o problema 3 e
	28 para o 4 (embora não esteja informado na tabela \ref{tab:resultados}),
	observa-se que a convergência se dá muito mais rapidamente para o primeiro,
	e é possível ver o ponto em que atinge a região viável, na qual vale a condição
	de parada (dentro da tolerância).

	Para o segundo problema, porém, o algoritmo
	acaba não alcançando a região viável, pois, conforme ele se aproxima dela,
	os gradientes das restrições se tornam linearmente dependentes e ortogonais
	a $\nabla f$ , de forma que
	não seja possível anular $\nabla \mathcal{L}$.
